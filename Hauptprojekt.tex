\documentclass{llncs}
\usepackage[ngerman]{babel}
\usepackage[utf8]{inputenc}
\usepackage{listings}
\usepackage{graphicx}
\usepackage{cite}
\usepackage{url}
\usepackage{natbib}


\title{Hauptprojekt}
\author{Alexander Piehl\\\email{alexander.piehl@haw-hamburg.de}
\institute{Hamburg University of Applied Sciences,\\Dept. Computer Science, \\ Berliner Tor 7\\ 20099 Hamburg, Germany\\}}

\begin{document}
\maketitle
\section{Einleitung}
\nocite{*}
\section{REST}
REST ist ein Architekturstil für verteilte Systeme wie unter anderem Anwendungen, die nach dem Schema Client-Server arbeiten.
Dabei gewann REST eine sehr große Beliebtheit für die Entwicklung von Webservices, da REST Webservices wohl nicht nur leichter zu implementieren sind, sondern auch einfacher zu skalieren sind. \cite{chakrabarti2009test}. 
Unter Anderem aus diesen Gründen stellten Google, Facebook und Yahoo ihre Services von SOAP auf REST um \cite{rodriguez2008restful, navas2014rest}.

Die Abkürzung REST  steht für Representational State Transfer \cite{chakrabarti2009test}. Aus dem Namen kann man ableiten, dass der gegenständliche Zustand übertragen wird. Dies ist ein Hauptmerkmal von REST. Denn bei REST 
\subsection{Besonderheiten beim Testen}
\section{Schnittstellen Tests}
\subsection{Consumer Driven Contract Test}
\section{Fazit}


\bibliography{literatur}
\bibliographystyle{alpha}

\end{document}