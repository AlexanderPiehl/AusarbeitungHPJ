\documentclass{llncs}
\usepackage[ngerman]{babel}
\usepackage[utf8]{inputenc}
\usepackage{listings}
\usepackage{graphicx}
\usepackage{cite}
\usepackage{url}
\usepackage{natbib}


\title{Hauptprojekt}
\author{Alexander Piehl\\\email{alexander.piehl@haw-hamburg.de}
\institute{Hamburg University of Applied Sciences,\\Dept. Computer Science, \\ Berliner Tor 7\\ 20099 Hamburg, Germany\\}}

\begin{document}
\maketitle
\section{Einleitung}

\section{REST}
REST ist ein Architekturstil für verteilte Systeme wie unter anderem Anwendung, die nach dem Schema Client-Server arbeiten.
Dabei steht REST für Representational State Transfer \cite{chakrabarti2009test}. 
\subsection{Besonderheiten beim Testen}
\section{Schnittstellen Tests}
\subsection{Consumer Driven Contract Test}
\section{Fazit}


\bibliography{literatur.bib}
\bibliographystyle{alpha}

\end{document}